%*************************************************
% In this file the abstract is typeset.
% Make changes accordingly.
%*************************************************

\addcontentsline{toc}{section}{چکیده}
%\newgeometry{left=2.5cm,right=3cm,top=3cm,bottom=2.5cm,includehead=false,headsep=1cm,footnotesep=.5cm}
\newgeometry{left=2cm,right=2cm,top=3.5cm,bottom=3cm,includehead=false,headsep=.5cm,footnotesep=.2cm}
\setcounter{page}{1}
\pagenumbering{arabic}						% شماره صفحات با عدد
\thispagestyle{empty}

%~\vfill

\subsection*{چکیده}
\begin{small}
    \baselineskip=0.7cm
    جمع آوری داده و پرچسب زدن داده‌ها از مهم‌ترین مراحل پیش‌پردازش در یادگیری ماشین است.اما همواره جمع‌آوری به سادگی نیست و ما به داده‌های برچسب‌خورده دسترسی نداریم. عدم دسترسی به دادگان مناسب، هزینه زیاد جمع آوری دادگان و مشاهده نمونه‌های جدید از عوامل حرکت به سمت یادگیری با نمونه کم، یک نمونه و در نهایت بدون نمونه بوده‌اند.
    \\
    در سیستم‌های تشخیص چهره، یکی از ارکان اساسی برای سلامت و کارایی سیستم توانایی جلوگیری از حملات مختلف و تشخیص جعل تصویر است. راهکارها و روش‌های بسیاری در طول سالیان مخنلف ارائه شده است؛ اما نکته حائز اهمیت تغییر کردن و به‌روز شدن حملات است که باعث می‌شود این کشمکش بین حملات و سیستم‌های تشخیص حملات و جعل تصاویر همواره وجود داشته باشد.
    \\
    از روش‌های پر کاربردی که برای تولید تصاویر جعلی و انجام حملات به سیستم‌های تشخیص چهره استفاده می‌شود روش‌های تخاصمی و استفاده از شبکه‌های متولد متخاصم است. جلوگیری از این حملات و نیازمند روش‌های هوشمند و به‌روزی است به همین به سراغ استفاده از روش یادگیری بدون نمونه رفته‌ایم تا بتوانیم با استفاده از آن از حملات جلوگیری کنیم و جعل تصویر چهره را تشخیص دهیم.
%\begin{LTRitems}
% few-shot learning، one-shot learning ،zero-shot learning و meta learning
% \end{LTRitems}
%\lr{ few-shot learning, one-shot learning, zero-shot learning, meta learning}

    \vspace*{0.5 cm}

    \noindent\textbf{واژه‌های کلیدی:}
    ۱-یادگیری ماشین  ۲-جعل تصویر چهره ۳-شبکه‌های مولد تخاصمی ۴-یادگیری بدون نمونه
\end{small}