\chapter{مقدمه}
\label{chap:introduction}
تشخیص چهره یکی از روش‌های
\trans{احراز هویت بیومتریک}{biometric authentication}
است که در قدیم توسط سیستم‌های امنیتی پیشرفته انجام می‌شد. در روش‌های بیومتریک از اثر انگشت یا کل دست، عنبیه چشم، صدا، چهره و غیره استفاده می‌شود که متداول‌ترین آن استفاده از چهره فرد است. امروزه با توجه به پیشرفت تکنولوژی و همه‌گیر شدن آن تشخیص چهره را در گوشی‌های هوشمند همراه می‌توان یافت.

همه گیری این تکنولوژی باعث پیشرفت‌های فراوان آن نیز شده است و امروزه گوشی‌های همراه می‌توانند به سرعت و با دقت بالا تشخیص چهره را انجام دهند و تقریبا به بهترین شکل اینکار را انجام می‌دهند. اما به امنیت این سیستم‌ها به اندازه کافی توجه نشده‌است. به این سیستم‌ها حملات متعددی صورت می‌گیرد که می‌توانند باعث خرابی سیستم، جعل هویت و یا گرفتن دسترسی کامل توسط حمله کننده و هکر شود. این سیستم‌ها به طور عمده امروزه بر اساس
\trans{شبکه‌های عصبی عمیق یا DNN}{Deep Neural Networks}
و
\trans{شبکه‌های عصبی کانولوشن یا CNN}{Convolutional Neural Network}
ساخته می‌شوند. در مقالات زیادی انواع حملات به این نوع شبکه‌ها بررسی شده‌اند.

تشخیص حملات و مقابله با آن‌ها یکی از مواردی است که باید به آن توجه ویژه‌ای داشت. یکی از مشکلات اساسی در حملات و جعل تصاویر صورت گرفته دسته بندی حملات و تشخیص نوع حمله است که به آسانی توسط انسان امکان‌پذیر نیست.
\section{صورت مسئله}
تشخیص و دسته‌بندی حملات به سیستم‌های تشخیص چهره و جعل‌ تصاویر چهره از بزرگ‌ترین معزلات توسعه دهندگان این سیستم‌هاست. تضمین امنیت سیستم برای جلوگیری از دور زدن سیستم، گرفتن دسترسی‌های بیشتر و یا احراز هویت به جای فرد دیگری توسط هکرها از مواردی است که  توسعه دهندگان باید مد نظر داشته باشند. با توجه به اینکه به طور عمده حملات در دسته‌های مشخصی قرار می‌گیرند و ویژگی‌هایی دارند که می‌توان آن‌ها را دسته‌بندی کرد استفاده از روش‌های دسته‌بندی به شیوه‌های مختلف می‌تواند کارآمد و کمک‌کننده باشد.

برای دسته‌بندی باید بتوانیم ویژگی‌های حملات مختلف را تشخیص دهیم. تشخیص این ویژگی‌ها به سادگی تشخیص ویژگی‌های تصاویر حیوانات مختلف برای دسته‌بندی آن‌ها نیست و ویژگی‌ها پیچیده‌تر و نیازمند سیستم‌های تشخیص دقیق‌تری هستند. بعد از تشخیص ویژگی‌های مورد نظر می‌توان حمله را دسته‌بندی کرد و متناسب با نوع حمله با آن مقابله کرد.

\section{اهداف پژوهش}
تجربه نشان داده است که یادگیری بدون نمونه برای دسته‌بندی و تشخیص نمونه‌هایی که کمتر تا کنون دیده شده‌اند یا اصلا دیده نشده‌اند، موفق عمل کرده و توانسته دسته‌بندی را به نحو احسنت انجام دهد. استفاده از یادگیری بدون نمونه در مواردی که پیدا کردن ویژگی‌های مشترک مانند حیوانات و اشیاء پیرامون به سادگی نیست و نمی‌توان فهرستی از ویژگی‌های مشترک را به آسانی تهیه کرد؛ موفق عمل کرده است. هدف پژوهش استفاده از دادگان‌های موجود و پر کاربرد در حوزه جعل تصویر چهره و یادگیری بدون نمونه برای ارائه راهکاری که قابلیت تشخیص و دسته‌بندی حملات مختلف به سیستم‌های تشخیص چهره را با دقتی نزدیک به دقت تشخیص چهره دارد، است.

\section{نوآوری پژوهش}
\section{ساختار پایان‌نامه}
در فصل دوم، به بررسی اجمالی درباره یادگیری بدون نمونه و انواع آن  می‌پردازیم و همچنین حملات و جعل‌های مختلف تصویر چهره را بررسی می‌کنیم. در فصل سوم، کارهای مرتبط به این پژوهش بررسی می‌شوند و مزایا و معایب آن‌ها به طور مختصر بررسی میگردد. در فصل چهارم، به تفضیل روش پیشنهادی و ایده اصلی بیان شده است. نحوه پیاده‌سازی و ساختار مدل بررسی شده‌است. در فصل ششم نتایج و ارزیابی‌های مرتبط بررسی شده و میزان کارایی و نتایج به‌دست آمده مورد تحلیل قرار گرفته‌است و در نهایت در فصل هفتم جمع‌بندی پایانی صورت گرفته‌است.
